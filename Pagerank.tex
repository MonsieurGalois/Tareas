\documentclass[letterpaper,11pt]{article}
\usepackage{amsfonts}
\usepackage{amsmath}
\usepackage{amsthm}
\usepackage{amssymb}
\usepackage{enumerate}
\usepackage{float}
\usepackage[top=2cm, bottom=2cm, left=2cm, right=2cm]{geometry} 
\usepackage{graphicx}
\usepackage{wrapfig}
\usepackage{url}
\usepackage[Spanish]{babel}
\usepackage[utf8]{inputenc}
\decimalpoint

\renewcommand{\quote}[1]{\lq\lq#1\rq\rq}

\begin{document}

\title {Análisis de algoritmos de búsqueda}
\date{Diciembre de 2013}
\maketitle

\subsection*{Objetivo}

Dar a conocer en qué consiste el PageRank de Google. Interpretar y traducir el Método de Google Pagerank del código de Matlab al código de Python.

\subsection*{Introducción}

PageRank es una marca registrada y patentada  por Google el 9 de enero de 1999 basado en ciertos algoritmos utilizados para asignar de forma numérica la relevancia de los documentos (o páginas web) indexados por un motor de búsqueda. El sistema PageRank es utilizado por el popular motor de búsqueda Google, con el objetivo de valorar y diferenciar las páginas web de acuerdo a su importancia.Fue desarrollado por los fundadores de Google, Larry Page (apellido del cual recibe el nombre este algoritmo) y Sergey Brin, en la Universidad de Stanford.\\
PageRank utiliza su vasta estructura de enlaces como un indicador del valor de una página en concreto. Google interpreta un enlace de una página A a una página B como un voto de la página A, para la página B. Pero Google mira más allá del volumen de votos o enlaces que una página recibe: Una página que está enlazada por muchas páginas con un PageRank alto consigue también un PageRank alto. Si no hay enlaces a una página web, no hay apoyo a esa página específica. El PageRank de la barra de Google va de 0 a 1.\\
\quote{1}  es el máximo PageRank posible y son muy pocos los sitios que gozan de esta calificación, \quote{.1} es la calificación mínima que recibe un sitio normal, y \quote{0} significa que el sitio ha sido penalizado o aún no ha recibido una calificación de PageRank. El PageRank de un sitio web puede ser calculado mediante un algoritmo basado en la aplicación de matrices, el cual detallaremos a continuación.


\subsection*{Desarrollo}

Sea $W$ el conjunto de $n$ páginas web que pueden ser visitadas mediante una cadena de enlace que comienzan desde una \textit{página raíz}. Sea $G=\{ g_{ij}\}$ una matriz de $n\times n$ a la que llamaremos \textit{matriz de conectividad}. Ésta matriz tiene elementos del tipo $\{ g_{ij}\}=1$ si hay un enlace de la página $j$ a la página $i$ y $\{ g_{ij}\}=0$ en caso contrario. La $j$-ésima columna muestra los enlaces que tiene la página $j$ hacia otras páginas. La suma de los números distintos de cero de la matriz $G$, es el número total de enlaces en $W$.\\
\newpage
Una matriz es naturalmente muy eficiente para almacenar datos en la computadora. Observemos que $G$ es una matriz dispersa\footnote{Una \emph{matriz dispersa} es una matriz de gran tamaño en la que la mayoría de sus elementos son cero.}. \\

Una matriz de este tipo es de gran utilidad en algoritmos computacionales, ya que:
\begin{itemize}
\item Es económica en cuanto a la ocupación de memoria, ya que mediante distintos formatos en su representación, los elementos nulos dejan de ocupar memoria.
\item El manejo de éstas es sencillo debido a su mayoría de ceros, lo cual repercute en la facilidad de las operaciones.
\item Economiza tiempo, ya que se usa menos el procesador de la computadora.
\end{itemize} 


Regresando al análisis de $G$, definamos $r_i$ y $c_j$ como la suma de los elementos del renglón $i$ y de la columna $j$, respectivamente, es decir:\\


$r_i = \displaystyle \sum _j g_{ij}$, $c_j = \displaystyle \sum _i g_{ij}$\\

Sea $p$ la probabilidad de que un usuario continúe pulsando \textit{links} al navegar por Internet en vez de escribir una \textit{url} directamente en la barra de direcciones o pulsar uno de sus marcadores. El valor estándar de $p$ es de 0.85, establecido por \textit{Google}. Entonces $1-p$ es la probabilidad de que el usuario deje de pulsar links y navegue directamente a otra web aleatoria. Por lo tanto $\delta = \displaystyle \frac{1-p}{n} $ representa la probabilidad de que una página en particular\footnote{Recordemos \emph{n} es el número total de páginas en $W$}, elegida aleatoriamente, sea visitada directamente.\\

Sea $A$ una matriz de $n \times n$ cuyos elementos son de la forma:\\

$a_{ij} = \left\{ \begin{array}{rcl}
pg_{ij}/c_j + \delta & : & c_j \neq 0\\
1/n & : & c_j = 0\\
\end{array}\right.$\\

Nótese que $A$ surge de escalonar $G$ por sus sumas de columas, es decir, depende de $c_j$. \\

La $j$-ésima columa (la suma de sus elementos) es la probabilidad de pasar de la página $j$ a otras páginas en la web, a través de enlaces. Si la página $j$ no tiene enlaces, entonces asignamos una probabilidad uniforme de $1/n$ a todos los elementos de tal columna.\\

Observemos que la mayoría de los elementos de $A$ valen $\delta$, la probabilidad de que una página salte a otra sin seguir enlaces. Ésto es porque al ser $G$ dispersa, la mayoría de sus elementos son cero.\\

La matriz $A$ es la matriz de probabilidad de transición de la cadena de Markov. Todos sus elementos toman valores estrictamente entre cero y uno, y la suma de cada columna es uno. \\

\newpage

Por la teoría de matrices, concluimos que existe una solución única distinta de cero a la ecuación:\\
\begin{equation}\tag{*}
	x = Ax
\end{equation}

Si este factor es elegido de tal manera que:\\
\begin{equation*}
	\displaystyle \sum _i x_i = 1 
\end{equation*}

Entonces $x$ es el \textit{vector estacionario} de la cadena de Markov y también es el \textit{PageRank} de Google. \\

Nótese que la ecuación (*) también puede verse como:

$$(I-A)x=0$$

Es decir, $x$ está en el espacio nulo de este sistema homogéneo.\\

En el siguiente ejemplo, veremos cómo obtener la matriz de conectividad de un conjunto de enlaces particular.\\

Sean $ W= \{ alpha,\ beta,\ gamma,\ delta,\ rho,\ sigma\}$ (donde la k-ésima página es la que aparece en el k-ésimo índice del conjunto) el conjunto de  páginas web con la siguiente relación:\\

\begin{figure}[H]
  \centering
    \includegraphics[scale=.7]{relacion}
  \caption{Relación de enlaces entre las páginas}
  \label{fig:ejemplo}
\end{figure}

Donde la k-ésima página es la que aparece en el k-ésimo índice del conjunto.\\

\newpage
Podemos generar la matriz de conectividad especificando los pares de índices $(i,j)$ con elementos distintos de cero. Por ejemplo, como \emph{alpha} se relaciona con \emph{beta}, entonces el par $(2,1)$ (la entrada $g_{21}$) es distinto de cero. Las 9 conexiones están descritas por:
\quad\\

$i = [2\ 6\ 3\ 4\ 4\ 5\ 6\ 1\ 1]$\\

 $j = [1\ 1\ 2\ 2\ 3\ 3\ 3\ 4\ 6]$\\

Luego, la matriz dispersa $G$, queda:


\[
\left(
\begin{array}{c c c c c c}
0&0&0&1&0&1\\
1&0&0&0&0&0\\
0&1&0&0&0&0\\
0&1&1&0&0&0\\
0&0&1&0&0&0\\
1&0&1&0&0&0\\
\end{array}
\right)
\]

\bigskip

entonces, $c_j$ para $j = 1,2,\cdots 6$, respectivamente, es:\\
\begin{center}
$\begin{array}{c c c c c c c}
c = 2&2&3&1&0&1\\
\end{array}$

\end{center}

Usando el valor estándar de $p=0.85$ y recordando que $\delta = (1-p)/n = 0.15/6 = 0.025$, la matriz $A$ correspondiente queda:\\

\[
\left(
\begin{array}{c c c c c c}
0.025&0.025&0.025&0.8750&0.1667&0.8750\\
0.4500&0.025&0.025&0.025&0.1667&0.025\\
0.025&0.4500&0.025&0.025&0.1667&0.025\\
0.025&0.4500&0.3083&0.025&0.1667&0.025\\
0.025&0.025&0.3083&0.025&0.1667&0.025\\
0.4500&0.025&0.3083&0.025&0.1667&0.025\\
\end{array}
\right)
\]

\bigskip

Resolviendo $(*)$ con el código en \textit{Python}, obtenemos el siguiente vector estacionario:\\
\begin{equation*}
x = \left( \begin{array}{c} 0.3210\\0.1705\\0.1066\\0.1368\\0.0643\\0.2007 \end{array} \right)
\end{equation*}

\newpage
 
\subsection*{Conclusión}

Finalmente, analicemos los resultados:
\[
\begin{tabular}{| c | c | c | c | c |}
\hline
{ }&{Pagerank}&{Enlances de entrada}&{Enlaces de salida} & {url}\\
\hline
{1}&{0.3210}&{2}&{2}&{http://www.alpha.com} \\
\hline
{6}&{0.2007}&{2}&{1}&{http://www.sigma.com}\\
\hline
{2}&{0.1705}&{1}&{2}&{http://www.beta.com}\\
\hline
{4}&{0.1368}&{2}&{1}&{http://www.delta.com}\\
\hline
{3}&{0.1066}&{1}&{3}&{http://www.gamma.com}\\
\hline
{5}&{0.0643}&{1}&{0}&{http://www.rho.com}\\
\hline
\end{tabular}
\]

\bigskip

Como se puede observar, la página \emph{alpha} es la que tiene mayor PageRank, es decir, tiene mayor número de votos y es la más visitada entre las 6 páginas analizadas.

\subsection*{Referencias}
\begin{itemize}

\item \url{ http://www.page-rank.es/algorit.html }
\item \url{ http://www.mathworks.com/moler/exm/chapters/pagerank.pdf}
\item \url{ http://www.fing.edu.uy/inco/cursos/comp1/teorico/2011/clase_23_2011_matDispersas.pdf}
\item \url{ http://es.wikipedia.org/wiki/PageRank} 


\end{itemize}



\end{document}